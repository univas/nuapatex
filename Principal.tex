%Documento principal do latex para uso no TCC do curso de Sistemas de Informação da Univás.
%Os alunos devem editar apenas os arquivos que estão no diretório "editaveis".
%As figuras que forem utilizadas no TCC devem ser colocadas na pasta "imagens".

%Atenção: Os alunos não devem editar os outros arquivos. Qualquer dúvida procure o professor de TCC e Desenvolvimento de Projetos

\documentclass[a4paper, 12pt, chapter=TITLE, oneside, english, brazil,oldfontcommands]{abntex2}

\usepackage{styles/CodeStyle}           %Formatação de códigos e listagens
\usepackage{styles/EventFlowStyle} %Estilo para o quadro de fluxo de eventos
\usepackage{styles/NuapaStyle}         %Estilo exigido pela Univás

\begin{document} %Início do documento

\pretextual %Início dos Elementos Pré-Textuais

\include{fixos/01_Capa}
\input{fixos/02_FolhaRosto}
\input{fixos/03_FichaCatalografica}
\input{fixos/04_FolhaAprovacao}
\input{editaveis/05_Dedicatoria}
\input{editaveis/06_Agradecimentos}
\begin{epigrafe}
\vspace*{\fill}
\begin{flushright}
\textit{‘‘Complicar é simples, \\
simplificar que é complicado.\\
(Paulo Sérgio dos Santos)\\
Atenção: \\%não pode haver quebra de linha sem nenhum texto nela
Os alunos devem colocar a frase que desejarem!}
\end{flushright}
\end{epigrafe}

\input{editaveis/08_Resumo}
\input{editaveis/09_Abstract}
\input{fixos/0A_ListasAutomaticas}
\input{editaveis/0B_Siglas}
\input{fixos/0C_Sumario}


\textual %Início dos Elementos Textuais

\chapter{INTRODUÇÃO}

\par Exemplo de uma nova referência bibliogáfia inserida para teste \cite{grafosRoberto2013}.


\par Exemplo simples de parágrafo utilizando o comando \texttt{$\backslash$par}. Pode-se utilizar (in\-cons\-titucional (exemplo de forçar separação de sílabas)).

\par Exemplo referenciando novamente, conforme mostrado pelo aulor \citeonline{grafosRoberto2013}.

\begin{citacao}
\textit{For five days over Easter from 1 April, Germans would have been asked to stay at home and reduce social contact. In-person religious services would have been cancelled, large family gatherings banned and almost all shops would have been closed} \cite{noticiaAlemanhaBBC}.
\end{citacao}

\par O \LaTeX~faz a ifenização automática, porém existem casos que é necessário forçá-lo. Veja no parágrafo anterior como forçar a ifenização, na palavra: inconstitucional.

\par Existem várias formas de fazer referências. As duas formas mais comuns são: a primeira é assim: \citeonline{revista_patio_pedagoria_ar_livre}, e a outra é mostrada conforme \cite{ecocentro}.


\section{Um exemplo de sub capítulo}


\par Aqui está um exemplo de uma citação direta:

\subsection{Um exemplo de sub sub capítulo}

\begin{citacao}
``Um exemplo de citação longa longa longa longa longa longa longa longa longa longa longa longa longa longa longa longa longa longa longa longa longa longa longa longa longa longa longa longa longa longa longa longa longa longa longa''. \cite{gadotti2003boniteza}
\end{citacao}

\par Continuando a introdução, identifica-se vários \ldots

\par Para ativar/desativar a quebra de linha automática é bem simples: clique no menu Visualizar e, no segundo bloco de opções, clique em Quebrar linhas automaticamente.



\input{editaveis/11_QuadroTeorico}
\input{editaveis/12_QuadroMetodologico}
\input{editaveis/13_Resultados}
\input{editaveis/14_Conclusao}


\postextual %Início dos Elementos Pós-Textuais

%\citeoption{abnt-etal-cite=2}
\citeoption{ABNT-final}

\bibliography{biblio}               % insere as REFERÊNCIAS (arquivo biblio.bib)
\addcontentsline{toc}{chapter}{REFERÊNCIAS} % adiciona o título das referências no Sumário

\input{editaveis/20_Apendices}
\input{editaveis/30_Anexos}

\printindex

\end{document}