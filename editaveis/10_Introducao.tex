\chapter{INTRODUÇÃO}

\par Exemplo de uma nova referência bibliogáfia inserida para teste \cite{grafosRoberto2013}.


\par Exemplo simples de parágrafo utilizando o comando \texttt{$\backslash$par}. Pode-se utilizar (in\-cons\-titucional (exemplo de forçar separação de sílabas)).

\par Exemplo referenciando novamente, conforme mostrado pelo aulor \citeonline{grafosRoberto2013}.

\begin{citacao}
\textit{For five days over Easter from 1 April, Germans would have been asked to stay at home and reduce social contact. In-person religious services would have been cancelled, large family gatherings banned and almost all shops would have been closed} \cite{noticiaAlemanhaBBC}.
\end{citacao}

\par O \LaTeX~faz a ifenização automática, porém existem casos que é necessário forçá-lo. Veja no parágrafo anterior como forçar a ifenização, na palavra: inconstitucional.

\par Existem várias formas de fazer referências. As duas formas mais comuns são: a primeira é assim: \citeonline{revista_patio_pedagoria_ar_livre}, e a outra é mostrada conforme \cite{ecocentro}.


\section{Um exemplo de sub capítulo}


\par Aqui está um exemplo de uma citação direta:

\subsection{Um exemplo de sub sub capítulo}

\begin{citacao}
``Um exemplo de citação longa longa longa longa longa longa longa longa longa longa longa longa longa longa longa longa longa longa longa longa longa longa longa longa longa longa longa longa longa longa longa longa longa longa longa''. \cite{gadotti2003boniteza}
\end{citacao}

\par Continuando a introdução, identifica-se vários \ldots

\par Para ativar/desativar a quebra de linha automática é bem simples: clique no menu Visualizar e, no segundo bloco de opções, clique em Quebrar linhas automaticamente.


